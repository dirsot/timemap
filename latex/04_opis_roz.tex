\chapter{Opis rozwiązania}
\label{cha:Opis rozwiązania}

W poniższym rozdziale omówione zostaną kroki pracy.

\section{Transmisja danych}
\label{sec:transmisjaDanych}

Pierwszym aspektem który należy rozwiązać jest sposób przesyłania danych. Problem ten jest szczególnie istotny w omawianej pracy z uwagi na możliwość przesyłania informacji o granica lub innych liniach przezentowanych na mapie. Do opisu kwadratowego obszaru wymagane jest przesłanie informacji o 4 punktach. Jeżeli będziemy chcieli przekazać dokłądniejszy zarys obszaru, zaprezentować granicę państwa lub linię frotnu wojennego linia prosta w większości przypadków będzie zbyt ogólnym przybliżeniem, nie oddającym prawdziwej sytuacji.

Z raportu Akamai wynika że śrenia przepływność łączy internetowych dla użytkowników korzystających z puli adresów IP przeznaczonych dla Polski w I kwartale 2012 r. wynosiła 5Mb/s  \underline{\texttt{http://www.rp.pl/artykul/924483.html}} (dostęp 13.04.2014). Jest to bardzo dobry wynik któy plasuje Polskę w czołówce rankingu. Pomimo tego nie można pominąć faktu optymalizacji zapytać i danych przesyłanych, wymieniane dane pomiędzy użytkownikiem a serwerem powinny być jak najmniejsze. Duża popularność urządzeń mobilnych w których dostęp do internetu jest zapewniany często poprzez sieć bezprzewodową a dostęp do interentu nie jest jeszcze tak dogodny jak jest to w przypadku użytkowników stacjonarnych  wymusza optymalizację.

Kolejnym powodem dla którego odpowiedzi serwera powinny być jak najlżesze jest koszt pracy samego serwera. Jest to szczególnie widoczne w dużych aplikacjach mających wiele urzytkowników, czas jaki jest przeznaczany dla pojedyńczego użytkownika jest mnożony przez ich ilość. Z tego powodu zawsze podczas zwiększania ilości użtkowników korzystających z aplikacji następuje czas w którym należy zacząć korzystać z dodatkowego serwera. Celem programisty tworzącego kod który będzie wykorzystywał zasoby serwera(zarówno czas jak i pamięć) jest dbanie aby moment w którym niezbędne będzie korzystanie z większej ilośći maszym nastąpił przy jak największej ilości użytkowników.

xml - 729  557
json - 695  400
\cite{JsonXmlComp}
\cite{JsonXmlAjaxComp}

\subsection{XML}
\label{subsec:xml}
\lstset{language=XML}
\begin{lstlisting}[caption=caption]
<?xml version="1.0" ?>
<map>
	<title>Tove</title>
	<description>Jani</description>
	<from>123444</from>
	<to>123444</to>
	<markers>
		<marker>
			<x>123.344</x>
			<y>123.344</y>
			<z>123.344</z>
			<from>123444</from>
			<to>123444</to>
			<description>Jani</description>
			<icon>Jani</icon>
		</marker>
		<marker>
			<x>123.344</x>
			<y>123.344</y>
			<z>123.344</z>
			<from>123444</from>
			<to>123444</to>
			<description>Jani</description>
			<icon>Jani</icon>
		</marker>
		<marker>
			<x>123.344</x>
			<y>123.344</y>
			<z>123.344</z>
			<from>123444</from>
			<to>123444</to>
			<description>Jani</description>
			<icon>Jani</icon>
		</marker>
	</markers>
</map>
\end{lstlisting}

\subsection{JSON}
\label{subsec:json}

\lstset{language=JavaScript}
\begin{lstlisting}[caption=json]
{
    "title": "Tove",
    "description": "Jani",
    "from": 123444,
	"to": 123444,
    "markers": [
        {
            "x": "123.344",
			"y": "123.344",
			"z": "123.344",
			"from": "123444",
			"to": "123444",
            "description": "Jani"
            "icon": "Jani"
        },
		{
            "x": "123.344",
			"y": "123.344",
			"z": "123.344",
			"from": "123444",
			"to": "123444",
            "description": "Jani"
            "icon": "Jani"
        },
		{
            "x": "123.344",
			"y": "123.344",
			"z": "123.344",
			"from": "123444",
			"to": "123444",
            "description": "Jani"
            "icon": "Jani"
        },
    ]
}
\end{lstlisting}

\subsection{Przybliżanie wyników}
\label{subsec:przyblizanie}


\subsection{Skaner kml}
\label{subsec:scaner}



\section{Interferjs}
\label{sec:Interferjs}


