\section{Przechowywanie i transmisja danych}
\label{sec:przesyl}

Rozdział ten zawiera informacje związane z analizą wybranych metod przechowywania danych a także możliwości ich transmisji.

\subsection{GIS}
\label{subsec:gis}

GIS(Geographical Information Systems) jest to System Informacji Geograficznych, jest to zbiór wiedzy, informacji do pozyskiwania i analizowania danych przestrzennych. Dane są pobieranie w trakcie naziemnych pomiarów jak i zbierane przy użyciu systemu GPS. Informacje te są następnie zapisywane do określonych formatów,dzięki czemu możliwa jest ich dalsza analiza. \underline{\texttt{http://www.gis-support.pl/co-to-jest-gis/}}


\subsubsection{Wykorzystywany układ współrzędnych}
\label{subsec:uklad}

Istnieje wiele układów współrzędnych któe służą do opisu pojedyńczego punktu na powierzchni ziemi.Na stronie \underline{\texttt{http://www.spatialreference.org/ref/epsg/}} (dostęp) znajduje się lista zawierająca przykłady zapisu różnych obszarów kuli ziemskiej przy użyciu różnych formatów. Zawierają one niezbędne infomacje w konkretnym zapisie.

Wersją która ma za zadanie być ogólnoświatowym formatem jest WGS(World Geodetic System). Jego ostatnia wersja  WSG84 jest powszechnie używana w urządzeniach do nawigacji.
Tradycyjny zapis oparty jest na wykorzystaniu stopni, minut i sekunk, jest on obecnie wypierany przez nowszy,łatwiejszy do obliczeń komputerówych. Poniżej zaprezentowano jak wygląda zapis w starszym i nowszym punktu określającego położenie budynku A-0 uczelni AGH.

\begin{itemize}

\item
Pierwotny zapis

50°03'52.2803", 019°55'23.7968"
\item
Zapis unowocześniony

50.06452231874906, 19.923276901245117
\end{itemize}

Drugi zapis został wykorzystany do przechowywania danych w plikach, tak aby wymiana i wspólna praca była jak najprostsza. Wybór ten sprawił że nie występuje problem konwersji punktu do formatu czytelnego dla komputera, można go bez problemu odczaytać jako liczbę.

\subsection{Storage}
\label{subsec:storage5}
Aby stworzyć framework któy był by w stanie działać przy minimalnej konieczności konfiguracji dodatkowych środowisk postanowiono aby dane w pierwszej kolejności były przechowywane po stronie klienta. Do tego celu nadaje się funkcją stworzona w ostatniej wersji HTML którą jest Storage. Jest ona dokładniej omówiona w podręczniku do HTML5  \cite{html5dive}. Pozwala on na przechowywanie danych w przeglądarce użytkownika. Różnicą w stosunku do ciasteczek które również potrafią przechowywać informacje o konretnym użytkonwniku jest:
\begin{itemize}
\item
Większy rozmiar dostępnej pamięci m.in. Chrome 5MB \nocite{chrome5mb}, IE 10MB
\item
Informacje przechowywane są po stronie użytkownka, nie są przesyłane za każdym razem do serwera.
\item
Informacja może być przechowywana przez długi okres czasu.
\end{itemize}



Dodatkowo nie można pominąć faktu istnienia dwóch rodzaii tej pamięci.
\begin{itemize}

\item
Session Storage
Dane przechowywane są w kontekście sesji użytkownika, są one tracone w momencie zamknięcia okna przeglądarki.

\item
Local Storage
Teoretycznie dane są przechowywane w nieskończoność, do momentu kiedy użytkonwik nie usunie ich. Zamknięcię sesji nie powoduje usunięcia danych.

\end{itemize}

Powodem który wymaga wykorzystania tego typu pamięci jest możliwość przechwowywania danych nad którymi użytkonwik pracuje w aktualnym czasie. Nie potrzebuje on informacji które były dla niego istotne podczas poprzedniej wizyty,sesji. Sytuacja ta jednoznacznie wskazuje że lepszym wyborem jest wybór pamięci sesyjnej(Session Storage).


Wadą tej pamięci jest jej interfejs. Obecnie przechowywany sposób danych to mapowanie w postaci napis->napis. Wymusza to aby każde dane które chcemy przechować muszą być w formie ciągu znaków. Przykład \ref{lis:storage} przedstawia w jaki sposób możemy obiekt zawierający imię i nazwisko zapisać w pamięci. Linia 4 przedstawia obiekt w postaci której chcielibyśmy go przechować. Niestety zwykłe przypisanie do zmiennej w pamięci powoduje że jedynie typ instancji zostaje zapisany. Aby móc zapisać w poprawnej formie dane musimy doknać serializacji danych. Czynność tą możemy wykonać przy pomocy metody stringify z obiektu JSON, wynikiem jest ciąg znaków który możemy bez problemu zapisać w pamięci sesyjnej. Do odzyskania pierwotnego obiektu, odtworzenia go z zapisanego napisu wykorzystujemy metodę parse również z obiektu JSON.

\lstset{language=JavaScript}
\label{lis:storage}
\begin{lstlisting}[caption=json]
      uzytkownik={};
      uzytkonwik.imie='Jan'
      uzytkownik.nazwisko='Kowalski'
      //uzytkownik : Object {imie: "Jan", nazwisko: "Kowalski"}

      sessionStorage.u1 = uzytkownik
      //sessionStorage.u1 : "[object Object]"

      sessionStorage.u2 = JSON.stringify(uzytkownik)
      //sessionStorage.u2 : "{"imie":"Jan","nazwisko":"Kowalski"}"

      uzytkonwik2 = JSON.parse(sessionStorage.u2)
      //uzytkonwik2 : Object {imie: "Jan", nazwisko: "Kowalski"}
\end{lstlisting}



Wsparcie dla pamięci Storage nie jest obecne zazwyczaj w nowszych wersach przeglądarek. Na stronie \underline{\texttt{http://www.html5rocks.com/en/features/storage}} możemy sprawdzić aktualny stan większości przęglądarek.


\subsection{Relacyjne bazy danych}
\label{sec:relacyjne}

Początkowo zakładano wykorzystanie relacyjnej bazy danych do przechowywania danych.
Zdecydowaną się na MySqL(z uwagi na jej powrzechne wykorzystanie i brak opłat licencyjnych),jedną z table wypełniono testowymi danymi w ilości 1 miliona rekordów o strukturze zbioru punktów o określonym czasie trwania zawierała kolumny takie jak data początkowa i końcowa występowania, dwie współrzędne. W celach optymalizacyjnych koszt zapytań zastosowano indeks na pola po któych wykonywane będzię wyszukiwanie danych.
Stworzona została prosta aplikacja napisana w języku Python przy pomocy frameworka Django v.2.1, miałą ona na celu wysłanie wysłanie zapytania o konkretne punkty które zawierały się w okreśłonym przedziale czasie(ich czas trwania zawierał się w szukanym zakresie) a następnie otrzymane punkty rysowała na ekranie. Na tak przygotowanym środowisku przeprowadzono testy sprawdzające czas potrzebny na pobranie i wyświetleie danych.

Zadowalające wyniki otrzymywano gdy ilość elementów nie przekraczała 10 tysięcy punktów. Gdy zwiększono szukany zakres dat, tym samym zwiększono ilość punktów do wyświetlenia czas potrzebny do wyświetlenia danych wynosił ponad jedną sekundę. Z uwagi na dużą ilość danych które towarzyszą informacjom mapowym(docelowa baza będzie zawierała więcej danych a ilość jednorazowo wyciąganych informacji może być większa) zdecydowano się nie kontynuować prób z wykorzystaniem tego typu baz danych.

\subsection{NoSQL}
\label{sec:nosql}

W celu pernamentnego przechowywania i rozpowszechniania danych pomiędzy różnymi użytkownikami postanowiono wykorzystać nierelacyjną bazę danych jaką jest MongoDB. Posiada ona interfejsc służący do komunikacji z JavaScript-em, możliwe jest wykonywanie skryptów które bezpośrednio połączą się i uzyskają dane\cite{mongojs}. W omawianym projekcie nie jest to możliwe, nawet przy użyciu zapytań Ajax-owych nie będzie możliwe zapewnienie bezpieczeństwa danym pozwalającym na połączenie a w efekcjie edycję danych bezpośrednio w bazie danych.
Postanowiono aby aplikacji która będzie korzystała z tworzonego frawerork-a dostarczała metod zapewniających dostęp do bazy danych. Zachowanie takie spełnia założenia projektu, infromacje i wszelkie zmiany dokonane na nich są przechowywane w lokalnej pamięci z możliwością eksportu ich do pliku tekstowego bez konieczności ingerencji w aplikację nadrzęczną. W przypadku chęci posiadania błyskawicznej możliwości dzielenia się informacjami i dokonywanymi zmianami należy zapewnić interfejs do tego celu. Przykładową implementację takiego rozwiązania stworzono w języku PHP. Powstała strona która korzystała z opracowanego framework-u, dla celów komunikacji z bazą danych udostępniała adres ./resources/mongodb który na zapytanie przesłane w metodzie POST wykonywała zwracała dane dla podanych warunków. Adres ten dla odpowiednich danych potrafił dokonać zmian(edytować lub usunąć dane).
Ponieżej zaprezntowano listening prezentujący minimalne obiekty które nalezy wysłać na przygotowany adres aby wykonać odpowiednie akcje. Pierwszy ma za zadanie wykonać akcję select pobierający wszystkie dane należące do mapy od numerze "123" które zaszły w podanym zakresie czasu. Drugi obiekt odnosi się do akcji aktualizacji danych, zmienia położenia punktu o id równym 234,jego tytuł a także poziomy na których będzie on widoczny, funkcjonalnośc ta została opisana w sekcji  \ref{sec:wizualizacja}.

\lstset{language=JavaScript}
\begin{lstlisting}[caption=caption2]
Object {method: "select", mapId: "123", fromDate: "2013-01-01 12:30", toDate: "2013-01-31"}


Object{}
YCord: "19.9232769"
mapId: "123"
pointId: "234"
method: "update"
showFrom: "1"
showTo: "8"
title: "Lepszy tytuł"
xCord: "50.0645223187"

\end{lstlisting}



