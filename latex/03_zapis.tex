

\section{Przechowywanie i transmisja danych}
\label{sec:przesyl}

Rozdział ten zawiera informacje związane z analizą wybranych metod przechowywania danych a także możliwości ich transmisji.
Omawia próbę podjęcia pracy na relacyjnej bazie danych.

\subsection{Format danych}
\label{sec:format}
\section{Format danych}
\label{sec:dataformat}

W poprzednich punktach opisano metody przechwywania danych w pamięci komputera, w szeczólności pamięci tymczasowej którą dostarczają nowoczesne przeglądarki internetowe. Jednak aby móc korzystać z informacji stworzonych przez innego użytkownika, lub nawet pracować nad własnym projektem stworzonym we wcześniejszym czasie niezbednę jest trwałe zapisanie danych. W rozdziale \ref{subsec:htmluse} zostały omówione sposoby które oferują możliwość przchowywania danych, niestety wszystkie te rozwiązania posiadają wspólną wadę, jest to brak mobilności danych. Znajdują się one na dysku jednego komputera a ich przeniesienie na inny pomimo że nie jest nie możliwe, jest bardzo uciążliwe i nie praktyczne. Z tego powodu przeprowadzona została analiza aktualnie używanych formatów danych zapisywanych do zewnętrznych plików.

\subsection{GIS}
\label{subsec:gis}

GIS(Geographical Information Systems) jest to System Informacji Geograficznych, jest to zbiór wiedzy, informacji do pozyskiwania i analizowania danych przestrzennych. Dane są pobieranie w trakcie naziemnych pomiarów jak i zbierane przy użyciu systemu GPS. Informacje te są następnie zapisywane do określonych formatów,dzięki czemu możliwa jest ich dalsza analiza. \underline{\texttt{http://www.gis-support.pl/co-to-jest-gis/}}


\subsection{Wykorzystywany układ współrzędnych}
\label{subsec:uklad}

Istnieje wiele układów współrzędnych któe służą do opisu pojedyńczego punktu na powierzchni ziemi.Na stronie \underline{\texttt{http://www.spatialreference.org/ref/epsg/}} (dostęp) znajduje się lista zawierająca przykłady zapisu różnych obszarów kuli ziemskiej przy użyciu różnych formatów. Zawierają one niezbędne infomacje w konkretnym zapisie.

Wersją która ma za zadanie być ogólnoświatowym formatem jest WGS(World Geodetic System). Jego ostatnia wersja  WSG84 jest powszechnie używana w urządzeniach do nawigacji.
Tradycyjny zapis oparty jest na wykorzystaniu stopni, minut i sekunk, jest on obecnie wypierany przez nowszy,łatwiejszy do obliczeń komputerówych. Poniżej zaprezentowano jak wygląda zapis w starszym i nowszym punktu określającego położenie budynku A-0 uczelni AGH.

\begin{itemize}

\item
Pierwotny zapis

50°03'52.2803", 019°55'23.7968"
\item
Zapis unowocześniony

50.06452231874906, 19.923276901245117
\end{itemize}

Drugi zapis został wykorzystany do przechowywania danych w plikach, tak aby wymiana i wspólna praca była jak najprostsza. Wybór ten sprawił że nie występuje problem konwersji punktu do formatu czytelnego dla komputera, można go bez problemu odczaytać jako liczbę.

\subsection{Relacyjne bazy danych}
\label{sec:relacyjne}

\subsection{NoSQL}
\label{sec:nosql}

\subsection{Pamięć lokalna}
\label{sec:localStorage}

\subsection{Wykorzystanie HTML5}
\label{subsec:htmluse}

Wstępne założenia zakładały wykorzystanie możliwości rysowania kształtów przy użyciu ogólnodostępnych map. Rozważania w sekcji \ref{sec:Rodzaje map} wskazały że najlepszym wyborem są Google Maps. Niestety opcja rysowania kształtów któa dostarcza HTML5 wymaga uzycia znacznika canvas, nie jest to kompatybilne z metodą wykorzystywania maps dostarczanych od omawianej firmy która wymaga wykorzystanie znacznika div.W momencie tworzenia tej pracy nie ma możliwości współpracy wybranych map z technologią rysowania kształtów geometrycznych. Sytuacja taka wymusza wykorzystanie innego rozwiązania do tworzenia interesujących nas warstw.

\subsection{Storage}
\label{subsec:storage5}
Inną ciekawą funkcją omówioną w podręczniku do HTML5  \cite{html5dive} jest Storage. Pozwala on na przechowywanie danych w przeglądarce użytkownika. Różnicą w stosunku do ciasteczek które również potrafią przechowywać informacje o konretnym użytkonwniku jest:
\begin{itemize}
\item
Większy rozmiar dostępnej pamięci m.in. Chrome 5MB \nocite{chrome5mb}, IE 10MB
\item
Informacje przechowywane są po stronie użytkownka, nie są przesyłane za każdym razem do serwera.
\item
Informacja może być przechowywana przez długi okres czasu.
\end{itemize}



Dodatkowo nie można pominąć faktu istnienia dwóch rodzaii tej pamięci.
\begin{itemize}

\item
Session Storage
Dane przechowywane są w kontekście sesji użytkownika, są one tracone w momencie zamknięcia okna przeglądarki.

\item
Local Storage
Teoretycznie dane są przechowywane w nieskończoność, do momentu kiedy użytkonwik nie usunie ich. Zamknięcię sesji nie powoduje usunięcia danych.

\end{itemize}

Powodem który wymaga wykorzystania tego typu pamięci jest możliwość przechwowywania danych nad którymi użytkonwik pracuje w aktualnym czasie. Nie potrzebuje on informacji które były dla niego istotne podczas poprzedniej wizyty,sesji. Sytuacja ta jednoznacznie wskazuje że lepszym wyborem jest wybór pamięci sesyjnej(Session Storage).


Wadą tej pamięci jest jej interfejs. Obecnie przechowywany sposób danych to mapowanie w postaci napis->napis. Wymusza to aby każde dane które chcemy przechować muszą być w formie ciągu znaków. Przykład \ref{lis:storage} przedstawia w jaki sposób możemy obiekt zawierający imię i nazwisko zapisać w pamięci. Linia 4 przedstawia obiekt w postaci której chcielibyśmy go przechować. Niestety zwykłe przypisanie do zmiennej w pamięci powoduje że jedynie typ instancji zostaje zapisany. Aby móc zapisać w poprawnej formie dane musimy doknać serializacji danych. Czynność tą możemy wykonać przy pomocy metody stringify z obiektu JSON, wynikiem jest ciąg znaków który możemy bez problemu zapisać w pamięci sesyjnej. Do odzyskania pierwotnego obiektu, odtworzenia go z zapisanego napisu wykorzystujemy metodę parse również z obiektu JSON.

\lstset{language=JavaScript}
\label{lis:storage}
\begin{lstlisting}[caption=json]
      uzytkownik={};
      uzytkonwik.imie='Jan'
      uzytkownik.nazwisko='Kowalski'
      //uzytkownik : Object {imie: "Jan", nazwisko: "Kowalski"}

      sessionStorage.u1 = uzytkownik
      //sessionStorage.u1 : "[object Object]"

      sessionStorage.u2 = JSON.stringify(uzytkownik)
      //sessionStorage.u2 : "{"imie":"Jan","nazwisko":"Kowalski"}"

      uzytkonwik2 = JSON.parse(sessionStorage.u2)
      //uzytkonwik2 : Object {imie: "Jan", nazwisko: "Kowalski"}
\end{lstlisting}



Wsparcie dla pamięci Storage nie jest obecne zazwyczaj w nowszych wersach przeglądarek. Na stronie \underline{\texttt{http://www.html5rocks.com/en/features/storage}} możemy sprawdzić aktualny stan większości przęglądarek.

\subsection{Web SQL Database}
\label{subsec:websql}

Koljenym ciekawym rozwiązaniem problemu przechowywania danych po stronie klienta jest Web Sql Database.
Jest to baza danych która została umieszczona po stronie klienta. Jest to dodatkowa warstwa która korzysta z SQlite.
Zaletą takiego rozwiązania jest przerzucenie kosztów obliczeniowych na stronę klienta, operacje takie jak sortowanie, filtrowanie na bazie danych wykonywane są przy wykorzystaniu zasobów komuptera klienta.

Nie wątpliwą zaletą w stosunku do omówionej w rozdziale \ref{subsec:storage5} pamięci jest możliwość ustalenia wielkości zalokowanej wielkości pamięci. Dzięki temu nie jesteśmy ograniczeni maksymalną dostępną ilością pamięci jak ma to miejsce w przypadku pamięci omawianej w poprzednim punkcie.

Przykład wykorzystania został zaprezentowany na \ref{lis:webSql}. Po stworzeniu bazy danych o nazwie "NowaBaza" i rozmiarze 2MB możemy korzystać z niej jak z każdej innej bazy danych. Możliwe są m.in. operacje tworzenia table(linie 3-5), wstawiania danych(parametry podajemy po treści zapytania, następnie możemy podać funkcje które zostaną wykonane w momencie sukcesu lub w przypadku wystąpienia błędu). Odczyt danych (linie 13-18) jest analogiczny do pozostałych operacji, dodatkowo pokazano w jaki sposób możliwa jest iteracja po otrzymanych wynikach.

Wadą tego rozwiązania jest jego popularność. Na stronie browser stats  możemy sprawdzić jaka jest popularność poszczególnych przeglądarek na świecie. Porównując te dane z dostępnością Web Sql dostępną na stronie \underline{\texttt{http://www.html5rocks.com/en/features/storage}} możemy wnioskować że prawie 45\% użytkowników( korzystających z IE i Firefox) nie mogą korzystać z tego rozwiązania, fakt ten dyskfalfikuje je.

\lstset{language=JavaScript}
\label{lis:webSql}
\begin{lstlisting}[caption=json]
      var db = window.openDatabase("NewDatabase", "1.0", "Description", 2 * 1024 * 1024);

      db.transaction(function(tx) {
        tx.executeSql("CREATE TABLE Person (id REAL UNIQUE, name TEXT)");
      });

      db.transaction(function(tx) {
        tx.executeSql("INSERT INTO Person (id, name) VALUES (?, ?)", [1, 'Jam'],
        onSuccess,
        onError);
                });

      db.transaction(function(tx) {
        tx.executeSql("SELECT * FROM Person", [], function(tx, result) {
            for (var i = 0, item = null; i < result.rows.length; i++) {
                ...
			}
        });
      });
\end{lstlisting}

\subsection{IndexedDB}
\label{subsec:indexDB}

Trzecim rozwiązaniem jest indexDb. Jest to szczególnie interesujące rozwiązanie gdyż wykorzystuje podejście nierelacyjnych baz danych(NoSQL).

Tradycyjne podejście do przechowywania danych zakłada wykorzystanie modelu relacyjnego do organizacji danych. Są one grupowane w relacje (zwane tablicami), każda z nich składa się z identycznych rekordów. Przykładem może być tablica przechowywujaca infromacje o osobie, mająca kolumny przeznaczone na dwa imiona i nazwisko. W takiej sytuacji wszyskie rekordy odnoszące się do osoby z jednym imieniem będą miały puste pole z drugim imieniem. Jest to sytuacja względnie nieproblematyczna, dopuszczenie pustych wartości ograniczy niepotrzebny wzrost wielkości bazy.
W przypadku chęci dodania nowej informacji, przykładowo daty urodzenia, spowoduje konieczość zmiany struktury bazy danych, dodanie kolumny do istniejącej tabeli lub stworzenie nowej dla dodatkowej wartości.

Kolejną wadą tego podejścia jest bardzo słaba skalowaność rozwiązania(możliwośc pracy na bardzo dużych danych). Korzystanie z wielu złączeń(czasami aby wykonać zapytanie i otrzymać interesujące nas informacje odwołujemy się do wielu tabel).
W zależności od poziomu izolacji transakcji możemy mieć dane które zostały zmienione w innej tranzakcji przd jej zakończeniem. Jeśli druga tranzakcja zostanie wycofana to w efekcie pobrane dane są nieaktualne.(Read uncommited)
Przeciwnym rozwiązaniem jest poziom Serializable w którym dane da którym rozpoczęto pracę są niedostępne dla innych tranzakcji, zapewnia to poprawność danych w każdej momencie jednak znacznie obniża wydajność bazy,szybkość odpowiedzi.

W chwilie pisania tego dokumentu jedynie 4 przeglądarki w najwyższej wersji wspierały tą technologię. Z tego powodu wykorzystanie to ogranicza docelową grupę odbiorców do wąskiego grona. 