\chapter{Podsumowanie}

Celem mojej pracy magisterskiej było stworzenie lekkiej i szybkiem aplikacji pozwalającej na łatwą prezentację zmian zachodzących na danym obszarze terytorialnym. Chciałem połączyć tradycyjne podejście do map, statycznej prezentacji treści z funkcjonalnością zmiany okresu czasu do któego informacje te się odnoszą tak aby w łatwy sposób możliwa byłaby prezentacja zmian zachodzących chociażby na polu walki, granic państw czy nawet zmian pogodowych.

Znaczna część pracy jest poświęcona analizie obecnie istniejącym technologiom pozwalającym na pracę z obrazami, przeglądzie nowoczesnych technologii. Zwrócono szczególną uwagę na możliwości dostarczane przez nową wersję języka HTML, zawiera on wiele opcji które wcześniej były możliwe w wyspecjalizowanych językach(np. wprowaszenie tagów określających plik audio i wideo). Ważne było aby nie próbować rozwiązywać problemów które juz wcześniej zostały rozwiązane. Mam nadzieję że analiza zamieszczona w pracy może stanowić wartościowy przegląd i zbiór przydatnych infromacji, nie tylko dla osób zainteresowanych rozojem aplikacji związanych z infromacjami geograficznymi ale także dla każdego chcącego zapoznać się z najnowszymi trendami w tworzeniu aplikacji mobilnych( HTML5,LESS).

Głównymi problemami które należało rozwiązać były związane z obsługą dużej ilości danych, przechowywaniem i prezenetacją. W początkowej fazie napotkano równiez problem wydajności, wyświetlenie kilkudziesięciu tysięcy punktów zajmowała kilka sekund. Jest to nieakceptowalna sytuacja, użytkownik który musiałby długo czekać na aktualizację podczas najmniejszej zmiany oglądanego przedziąłu czasu szybko zniehęciłby się, a sama aplikacja była by bezużyteczna. Wykorzystano najnowocześniejsze rozwiązania które pozwalają na otrzymanie bardzo dobrych wyników lecz stawiają pewne wymagania, głównie do wersji przeglądarki na któej działa aplikacja. Zdecydowano się na stosunkowo nową możliwość stworzenia bazy danych po stronie klienta, w jego przegladarce. Tradycyjne rozwiązania opierają się na utrzymywaniu danych po stronie serwera, po stronie użytkownika znajdowała się małą ilość informacji, głównie służąca do identyfikacji sesji zapisane w ciasteczkach(cookies).

Stworzona aplikacja spełnia minimalne warunki aby można było ją uznać za przydatną podczas procesu nauki. Posiada intuicyjny interejs pozwalający na łatwą obsługę, dostarcza cennych informacji a szeroki zakres prezentacji danych, możliwość tworzenia płynnych przejśc kształtów, kolorów sprawia że jest interesująca i przyciąga uwagę użytkownika. Wykorzystanie jej m.in. na lekcji histori pozwoli zmienić statyczne reprezentacje ruchów wojennych w ciekawe i interaktywne, taka forma może zwiększyć przyswajalność widzy i zachęcić do czestych powtórek poza salą lekcyjną. 
