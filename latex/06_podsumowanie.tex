\chapter{Podsumowanie}

Wysoki rozwój techonologi komputerowerj pozwala na redefiniowanie pojeć znanych człowiekowi od długiego czasu. Niniejsza praca udowania tą tezę, mapa która od zawsze była tylko niezmienną, graficzną repreznetacją terenu przy wykorzystaniu nowoczesnych technologi może stać się interatktywnym obiektem pozwalającym na ukazanie niedostępnych wcześniej danych w zupełnie nowy sposób.

Główna technologia wybrana do tworzenia aplikacji wykorzystywanej w przeglądarkach internetowych posiadającą możliwość interakcji użytkownika z danymi powinna dawać możliwość aktualizacji jedynie wybranych fragmentów danych, tak aby wyeliminować zbędne renderowanie stałych fragmentów. W takim przypadku dobrym wyborem jest JavaScript który dzięki założenion postawionym podczas jego tworzenia idealnie sprawdza się w tego typu zadaniach.

Format zapisu danych powinien zapewniać obsługę zawórno dowolnych punktów na kuli ziemskiej jak i dodatkowych elementów które mogą być dołączone go wejściowego zbioru danych. Ogólnodostępny format WGS pozwala na proste przechowywanie informacji o położeniu obiektów, jego zapis jest zarówno czytelny dla człowieka jak i dla komputera. Wybrany format zapisu plików, GML, pozwala na łatwe przechowaywanie niemal wszystkich informacji w jednym miejcu, jedynie obrazy graficzne wymagają podania ścieżki do ich fizycznej lokalizacji.

Wybór Google Maps jako podstawowe źródło informacji geograficznych, dostarczających statyczny obraz map dało dostęp do ogromnej bazy danych. Niestety najnowsze rozwiązania nie zawsze zapewniają kompatybilność ze starszymi, przykładem jest jeden z elementów HTML w wersji 5, canvas. Jego wykorzystanie okazało się niemowżliwe w połączeniu wybranymi mapami których interfejs został stworzony bez możliwości ich wykorzystania. Zaawansowaną grafikę udało się stworzyć przy pomocy SVG, formatu który nawet skompliwoany obraz potrafi zapisać w zwięzłej i krókiej formie oszczędzając miejsce na dysku przy zachowaniu szczegółowości i ilości detali. Stworzone algorytmy do stopniowego generowania obiektów i ich zmiany w zależniości od atrybutów mapy pozwalają na minimalizowanie wymagań w odniesiu do procesora i szybsze dostarczanie wartościowych danych.

Zwrócono również uwagę na bezpieczeństwo tworzonego projektu. Powstała aplikacja jest niezależna od jakichkolwiek zewnętrznych baz danych, dlatego ewentualna kontrola dostępu jest wykonywana przez nadrzędną wartstę którą jest serwis używający map. Kontrola poprawności danych jest możliwa dzięki wykorzystaniu funkcji tworzących sumę kontrolną, użytkownik może dokonać jej porównania z wartością do której autentyczność jest niezaprzeczalna. Taka czynność daje pewność że dane które posiada użytknownik są poprawne i stworzone przez autoryzowane osoby.

Zastosowane metody optymalizacji framework-u pozwoliły na zwięszenie płynności działania, jedynie niezbędne akcje są wykonywane podczas jego pracy. Operacje zbędnę, lub takie które mogą zostać wykonane w późniejszym czasie są zatrzymwane i uruchamiane jedy gdy jest to niezbędne.

Podsumowując stworzona aplikacja spełnia wszelkie warunki aby można było ją uznać za użyteczną podczas procesu nauki. Posiada intuicyjny interejs pozwalający na łatwą obsługę, dostarcza cennych informacji a szeroki zakres prezentacji danych, możliwość tworzenia płynnych przejśc kształtów, kolorów sprawia że jest interesująca i przyciąga uwagę użytkownika. Wykorzystanie jej m.in. na lekcji histori pozwoli zmienić statyczne reprezentacje ruchów wojennych w ciekawe i interaktywne, taka forma może zwiększyć przyswajalność widzy i zachęcić do czestych powtórek poza salą lekcyjną.

