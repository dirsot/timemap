\chapter{Wstęp}
\label{cha:wstep}

\nocite{gisSystems}
\nocite{webgis}
\nocite{imprxml}
\nocite{perfxml}


%---------------------------------------------------------------------------

\section{Temat pracy}
\label{sec:tematPracy}

Tematem pracy jest połączenie ogólnodostępnych źródeł informacji jakimi są mapy świata z możliwością zmiany czasu i co się z tym wiąże zmiany prezentowanych informacji. Tradycyjne podejście zakłada statyczną prezentację obrazu terenu. Są to często zdjęcia lotnicze które prezentują sytuację w konkretnym czasie. Poniższej pracy podjęto próby stworzenia aplikacji która bedzie rozszeżała tradycyjne podejście, doda czwarty wymiar, czas.

\section{Geneza teamatu}
\label{sec:geneza}

Mapy są znane człowiekowi od dawna, ich rozwój sprawił że obecnie, szczególnie ich wirtualne wersje zawierają dużą ilość dodatkowych opcji które mają za zadanie dostarczyć jeszcze więcej informacji. Niestety żadko można spotkać funkcjonalność która pozwalałaby na spojżenie w różne okresy czasu, zazwyczaj na jednej mapie znajduje się maksymalnie wąski okres czasu.

Obecny gwałtowny rozwój miast,zmian terytorialnych tworzy potrzebę prezentacji zachodzących zmian. Dodatkowym zastosowaniem może być prezentacja zmian pogodowych takich jak przesuwania się frontów atmosferycznych.

\section{Realizacja pracy}
\label{sec:realizacja}

Praca ma na celu stworzenie aplikacji dodającej do tradycyjnie znanych map dodatkowego wymiaru, czasu i możliwośći jego kontroli. Zadanie to wymaga rozwiązania kilku aspektów, są nimi.

\begin{itemize}

\item
Przechowywania i przysułu danych, docelowe zbiory danych mogą być duże. Należy zapewnić możliwość ich przechowywania w sposób przyjazny dla użytkownika.

\item
Wizualizacja danych, aplikacja powinna zapewniać estetyczną prezentację danych.

\item
Wiarygodność informacji, użytkownik powinien być pewien że informacje z których korzysta są poprawne i nie zmienione przez nieautoryzowane osoby.

\end{itemize} 

Każdemu z nich poświęcony jest odzielny rozdział aby w pełni przeanalizować tematykę.

Zdecydowano się na stworzenie frameworku, programu który dostarczy intuicyjny interfers do jego obsługi. Aby jego konfiguracja nie zajmowała czasu zrezygnowano z korzystania tradycyjnych baz danych, konieczność ich konfiguracji zniechęcała by nowych użytkowników. Wymaga to wykorzystania innego rozwiązania, które pozwalałoby na przechowywanie danych po stronie klienta, najlepiej w przeglądarkce klienta.

W początkowej fazie projektu rozważano możliwość współpracy osi czasu z możliwośćia tworzenia kształtó dostarczana przez HTML5, niestey brak kompatybilności tego rozwiązania z wybranymi mapami wymusił odrzucenie tej koncepcji.

\section{Zawartość pracy}
\label{sec:zawartoscPracy}


W rozdziale 2 przedstawiono krótki rys historyczny związany z obecnością map w histori człowieka. Rozdział 3 zawiera omówienie teoretycznych aspektów które zą związane z tematem pracy. Rozdziały 4 i 5 szczgółowo opisują implementację, napotkane problemy podczas tego procesu i sposób ich rozwiązania. Omówiono dlaczego jaki czynniki zdecydowały na wybrane konkretnych rozwiązań. Rozdział 6 zawiera krótki przykład wykorzystania stworzonej aplikacji. Ostatni rozdział jest podsumowaniem całej pracy, co udało się wykonać, co stanowiło barirę nie do pokonania a także zawiera informację o ewentualnych ścieżkach rozwoju.















