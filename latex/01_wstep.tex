\chapter{Wstęp}
\label{cha:wstep}

\nocite{gisSystems}
\nocite{webgis}
\nocite{imprxml}
\nocite{perfxml}


%---------------------------------------------------------------------------

\section{Temat pracy}
\label{sec:tematPracy}


Tematem pracy jest połączenie ogólnodostępnych źródeł informacji jakimi są mapy i dodanie do nich dodatkowego wymiaru, czau. Statyczny obraz zostanie rozszeżony o możliwość jego zmiany w zależności od aktualnie wybranego okresu. 

\section{Geneza teamatu}
\label{sec:geneza}

Mapy są znane człowiekowi od dawna, ich rozwój sprawił że obecnie, szczególnie ich wirtualne wersje zawierają dużą ilość dodatkowych opcji które mają za zadanie dostarczyć jeszcze więcej informacji. Niestety żadko można spotkać funkcjonalność która pozwalałaby na spojżenie w różne okresy czasu, zazwyczaj na jednej mapie znajduje się maksymalnie wąski okres czasu.

Obecny gwałtowny rozwój miast,zmian terytorialnych tworzy potrzebę prezentacji zachodzących zmian. Dodatkowym zastosowaniem może być prezentacja zmian pogodowych takich jak przesuwania się frontów atmosferycznych.

\section{Realizacja pracy}
\label{sec:realizacja}

Praca ma na celu stworzenie aplikacji dodającej do tradycyjnie znanych map dodatkowego wymiaru, czasu i możliwośći jego kontroli. Zdecydowano się na stworzenie frameworku, programu który spełni wszystkie wyamgania prjektu i dostarczy intuicyjny interfers do jego obsługi. Aby jego konfiguracja nie zajmowała czasu zrezygnowano z korzystania tradycyjnych baz danych, konieczność ich konfiguracji zniechęcała by nowych użytkowników. Praca jednak zawiera zbiór wskazań które serwerowa cześć aplikacji powinna spełniać aby możliwości stowrzonego progrmu zostały wykorzystane w pełni.

\section{Zawartość pracy}
\label{sec:zawartoscPracy}


W rozdziale 2 przedstawiono krótki rys historyczny związany z obecnością map w histori człowieka. Rozdział 3 zawiera omówienie teoretycznych aspektów które zą związane z tematem pracy. Rozdziały 4 i 5 szczgółowo opisują implementację, napotkane problemy podczas tego procesu i sposób ich rozwiązania. Omówiono dlaczego jaki czynniki zdecydowały na wybrane konkretnych rozwiązań. Rozdział 6 zawiera krótki przykład wykorzystania stworzonej aplikacji. Ostatni rozdział jest podsumowaniem całej pracy, co udało się wykonać, co stanowiło barirę nie do pokonania a także zawiera informację o ewentualnych ścieżkach rozwoju.















