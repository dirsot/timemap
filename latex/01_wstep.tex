\chapter{Wstęp}
\label{cha:wstep}

http://www.flashearth.com/

2-3 storny\\
wyraz $\tau$, $\epsilon$, $\chi$\\
Na stronie \underline{\texttt{http://kile.sourceforge.net/screenshots.php}}\\
%---------------------------------------------------------------------------

\section{Temat pracy}
\label{sec:tematPracy}

Tematem pracy jest połączenie ogólnodostępnych źródeł informacji jakimi są mapy świata z możliwością zmiany czasu i co się z tym wiąże zmiany prezentowanych informacji.

\section{Geneza teamatu}
\label{sec:geneza}

Mapy są znane człowiekowi od dawna, ich rozwój sprawił że obecnie, szczególnie ich wirtualne wersje zawierają dużą ilość dodatkowych opcji które mają za zadanie dostarczyć jeszcze więcej informacji. Niestety żadko można spotkać funkcjonalność która pozwalałaby na spojżenie w różne okresy czasu, zazwyczaj na jednej mapie znajduje się maksymalnie wąski okres czasu.

\section{Realizacja pracy}
\label{sec:realizacja}

Zdecydowano się na stworzenie frameworku, programu który będzie dostarczał intuicyjny interfers do jego obsługi. Aby jego konfiguracja nie zajmowała czasu zrezygnowano z korzystania tradycyjnych baz danych, konieczność ich konfiguracji zniechęcała by nowych użytkowników. Wymaga to wykorzystania innego rozwiązania, które pozwalałoby na przechowywanie danych po stronie klienta, najlepiej w przeglądarkce klienta.

W początkowej fazie projektu rozważano możliwość współpracy osi czasu z możliwośćia tworzenia kształtó dostarczana przez HTML5, niestey brak kompatybilności tego rozwiązania z wybranymi mapami wymusił odrzucenie tej koncepcji.

\section{Zawartość pracy}
\label{sec:zawartoscPracy}


W rozdziale 2 przedstawiono krótki rys historyczny związany z obecnością map w histori człowieka. Rozdział 3 zawiera omówienie teoretycznych aspektów które zą związane z tematem pracy. Rozdziały 4 i 5 szczgółowo opisują implementację, napotkane problemy podczas tego procesu i sposób ich rozwiązania. Omówiono dlaczego jaki czynniki zdecydowały na wybrane konkretnych rozwiązań. Rozdział 6 zawiera krótki przykład wykorzystania stworzonej aplikacji. Ostatni rozdział jest podsumowaniem całej pracy, co udało się wykonać, co stanowiło barirę nie do pokonania a także zawiera informację o ewentualnych ścieżkach rozwoju. 















