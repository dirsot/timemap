\chapter{Wstęp}
\label{cha:wstep}

\nocite{gisSystems}
\nocite{webgis}
\nocite{imprxml}
\nocite{perfxml}


%---------------------------------------------------------------------------

\section{Temat pracy}
\label{sec:tematPracy}

Celem pracy jest rozszeżenie tradycyjnych dwuwymiarowych obrazów kartograficznych i map o dodatkowy wymiar jakim jest czas.
Statyczny obraz zostanie rozszeżony o możliwość zmiany prezentowanych informacji w zależności od wybranego okresu. 
Zabieg ten ma na celu dostarczenie możliwości obserwacji zachodzących zmian pozwalając na lepszą wizualizację i 
przyswajanie widzy.

Większość aktualnie dostępnych map stanowi stałą reprezentację określającą wydarzenia na danym terenie. W tradycyjnej, 
drukowanej wersji nie jest możliwa żadna interakcja z danymi. Dzięki użyciu komputerów możliwe staje się umieszczenie 
większej ilości danych przy jednoczesnym zwiększeniu ich dokładności na jednym obiekcie. Dotychczasowa mapa zawierająca 
zbiór kilku lini określających granice terytorialne kraju znane z lekcji histori, może teraz zamienić się w przynną 
animację ukazującą dokładne zmiany w każdym punkcie w historii.



\section{Podstawowe cele}
\label{sec:geneza}

\begin{enumerate}
  \item Analiza dostępnych rozwiązań interaktywnych map
  \item Przegląd technologi i analiza problemów
  \begin{enumerate}
    \item Wybór narzędzi 
    \item Rozwiązanie wyzwań stawianych przez:
    \begin{itemize}
        \item Sposób zapisu i przechowywania danych
        \item Wizualizację różnego typu danych
        \item Bezpieczeństwo i wiarygoność danych
        \item Optymalizację i zapewnienie płynności działania
    \end{itemize}
  \end{enumerate}
  \item Imlementacja zaprojektowanej aplikacji
\end{enumerate}

\section{Zawartość pracy}
\label{sec:zawartoscPracy}


Pracę rozpoczyna Wstęp , w którym to został zawarty ogólny cel projektu. Posiada on również listę zadań które zosatły
opracowane i zrealizowane w trakcie jej tworzenia.

Drugi rozdział - Rys hisotryczny - zawiera któtką analizę dziejów i ewolucji map, od prymitywnych rysunków na ścianach aż 
po czasy współczesne.

Następny rozdział - Dostępne Rozwiązania - jest próbą analizy gotowych rozwiązań dostarcających połączenie statycznego obrazu 
i osi czasu. Jego zadaniem jest zapoznanie się z technologiami wykorzystywanymi w tego typu projektach i wybór najbardziej 
adekwatnych.

Rozdział - Ogólny opis rozwiązania - ma na celu przedstawienie biznesowej analizy, ogólny opis funkcjonalności które ma 
spełniać końcowy program. Przedstawienie granic systemu, jakie funkcje zawierają się w jego zakresie i co można przy jego 
pomocy wykonać.

Główna część - Opis rozwiązania - jest sprawozdaniem z prac jakie zostały przeprowadzone w procesie rozwiązywania napotkanych
problemów. Przedstawia sposób przechowywania danych umożliwaijący na błyskawiczne wykorzystanie aplikacji, ich prezentację 
z zależności od ich typu i ustawień programu. 

Następnie zawarty rozdział - Implementacja - to fragmenty kodu źródłowego który został stworzony podczas fazy tworzenia 
końcowego programu. Zawarte sekcje odnoszą się do najbardziej istotnych fragmentów kodu wraz z ich omówieniem.

Przedostani - Przykład użycia - przedstawia końcowy wynik otrzymany w procesie analizy danych wejściowych przez stworzony 
framework. 

Ostatni rozdział - Podsumowanie - ma on na celu analizę dokonać wykonanych w trakcie tworzenia omawianej pracy. 












