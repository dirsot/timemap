\documentclass[pdflatex,11pt]{aghdpl}
% \documentclass{aghdpl}               % przy kompilacji programem latex
% \documentclass[pdflatex,en]{aghdpl}  % praca w języku angielskim
\usepackage[polish]{babel}
\usepackage[utf8]{inputenc}

% dodatkowe pakiety
\usepackage{enumerate}
\usepackage{graphicx}% obraz
\usepackage{float}% polozenie obrazo
\usepackage{listings}% kod
\usepackage{color}

\lstloadlanguages{TeX,XML}
\usepackage{color}
\definecolor{gray}{rgb}{0.4,0.4,0.4}
\definecolor{darkblue}{rgb}{0.0,0.0,0.6}
\definecolor{cyan}{rgb}{0.0,0.6,0.6}
\definecolor{lightgray}{rgb}{.9,.9,.9}
\definecolor{darkgray}{rgb}{.4,.4,.4}
\definecolor{purple}{rgb}{0.65, 0.12, 0.82}

\lstset{
  columns=fullflexible,
  basicstyle=\tiny\sffamily,
  numbers=left,
  numberstyle=\tiny,
  showstringspaces=false,
  commentstyle=\color{gray}\upshape
}

\lstdefinelanguage{XML}
{
  morestring=[b]",
  morestring=[s]{>}{<},
  morecomment=[s]{<?}{?>},
  stringstyle=\color{black},
  identifierstyle=\color{darkblue},
  keywordstyle=\color{cyan},
  morekeywords={xmlns,version,marker,markers}% list your attributes here
}

\lstdefinelanguage{JavaScript}{
  keywords={typeof, new, true, false, catch, function, return, null, catch, switch, var, if, in, while, do, else, case, break},
  keywordstyle=\color{blue}\bfseries,
  ndkeywords={class, export, boolean, throw, implements, import, this},
  ndkeywordstyle=\color{darkgray}\bfseries,
  identifierstyle=\color{black},
  sensitive=false,
  comment=[l]{//},
  morecomment=[s]{/*}{*/},
  commentstyle=\color{purple}\ttfamily,
  stringstyle=\color{red}\ttfamily,
  morestring=[b]',
  morestring=[b]"
}
\lstset{
   language=JavaScript,
   backgroundcolor=\color{lightgray},
   extendedchars=true,
   basicstyle=\footnotesize\ttfamily,
   showstringspaces=false,
   showspaces=false,
   numbers=left,
   numberstyle=\footnotesize,
   numbersep=9pt,
   tabsize=2,
   breaklines=true,
   showtabs=false,
   captionpos=b
}

\lstset{
  literate={ą}{{\k{a}}}1
           {ć}{{\'c}}1
           {ę}{{\k{e}}}1
           {ó}{{\'o}}1
           {ń}{{\'n}}1
           {ł}{{\l{}}}1
           {ś}{{\'s}}1
           {ź}{{\'z}}1
           {ż}{{\.z}}1
           {Ą}{{\k{A}}}1
           {Ć}{{\'C}}1
           {Ę}{{\k{E}}}1
           {Ó}{{\'O}}1
           {Ń}{{\'N}}1
           {Ł}{{\L{}}}1
           {Ś}{{\'S}}1
           {Ź}{{\'Z}}1
           {Ż}{{\.Z}}1
}

%---------------------------------------------------------------------------

\author{Łukasz Zieńkowski}
\shortauthor{Ł. Zieńkowski}

\titlePL{Interaktywna mapa czasu z dodatkową osią czasu}
\titleEN{Thesis in \LaTeX}

\shorttitlePL{Interaktywna mapa czasu z dodatkową osią czasu}
\shorttitleEN{Thesis in \LaTeX}

\thesistypePL{Praca magisterska}
\thesistypeEN{Master of Science Thesis}

\supervisorPL{dr hab. Marcin Szpyrka}
\supervisorEN{Marcin Szpyrka Ph.D}

\date{2011}

\departmentPL{Katedra Automatyki}
\departmentEN{Department of Automatics}

\facultyPL{Wydział Elektrotechniki, Automatyki, Informatyki i Elektroniki}
\facultyEN{Faculty of Electrical Engineering, Automatics, Computer Science and Electronics}

\acknowledgements{Serdecznie dziękuję \dots tu ciąg dalszych podziękowań np. dla promotora, żony, sąsiada itp.}



\setlength{\cftsecnumwidth}{10mm}

%---------------------------------------------------------------------------

\begin{document}

\titlepages

\tableofcontents
\clearpage

\chapter{Wstęp}
\label{cha:wstep}

http://www.flashearth.com/

2-3 storny\\
wyraz $\tau$, $\epsilon$, $\chi$\\
W rodziale~\ref{cha:State of art}\\
obraz \ref{fig:lasVegas3}\\
w pliku \texttt{test.tex}.\\
Na stronie \underline{\texttt{http://kile.sourceforge.net/screenshots.php}}\\
%---------------------------------------------------------------------------

\section{Temat pracy}
\label{sec:tematPracy}

\section{Geneza teamatu}
\label{sec:geneza}

\section{Realizacja pracy}
\label{sec:realizacja}

\section{Zawartość pracy}
\label{sec:zawartoscPracy}




















\chapter{Rys historyczny}
\label{sec:hisotryMap}

Pomimo że w obecnych czasach, mapy nie są niczym niezwykłym, i wydają się codziennością nie zawsze tak było. W histori człowieka można znaleść okresy czasu kiedy kartografia była nieznana, lub była bardzo niedokładna.

Pierwsze malowidła które można uznać za graficzną reprezentację otoczenia archeolodzy napotkali w okolicach Pavloc (Czechy), datowaną są one na 25 wiek przed naszą erą \cite{pre2} . Na 14 wiek p.n.e datowane są wykopaliska w Navarre(Hiszpania) obejmujące rysunki na piaskowcu \cite{pre1}.

Jeszcze w XIV kartografia była bardzo ograniczona, mapy którymi posługiwano się nie były zbyt dokłądne. 12 października 1410 gdy Krzysztof Kolumb dotarł do brzegów wyspy San Salvador uznaj ją za jedną z wysp japońskich \cite{columb}. Aby lepiej zrozumieć powód tej pomyłki wystarczy spojrzeć na mapę stworzoną w XV wieku przez kartografa Henricusa Martellusa \ref{fig:worldMap1}. Widzimy na niej że m.in. obie ameryki nie były znane uwczesnym ludzion.

\begin{figure}[H]
  \centering
    \includegraphics[width=100mm]{ge/worldMap1.jpg}
  \caption{Mapa świata z 1489 roku.}
  \label{fig:worldMap1}
\end{figure}

\chapter{Podstawy teoretyczne}
\label{cha:podstawyTeoretyczne}

\section{Rodzaje map}
\label{sec:Rodzaje map}

Tworząc aplikację która ma dostarczać informacji korzystających z map należy zapoznać się dostępnymi źródłami. Z powodu szrokiego wyboru poniżej omówione zostaną jedynie aplikacje które dostarczają informacji ogólnoświatowych. Na polskim rynku dostępnych jest kilka rozwiązań, ich główną wadą jest ograniczenie do terytorium Polski, dodatkowo często nie dostarczają one obrazów satelitarnych, są to m.in. \underline{\texttt{http://zumi.pl}}


\subsection{Google Maps}
\label{subsec:Google Maps}
\nocite{googlemapsbegin}
Rysunek \ref{fig:googleMaps_1} przedstawia obraz otrzymany w aplikacji Google Maps. Dodatkowo włączona opcja prezentacji natężenia ruchu jedynie potwierdza duże możliwości i łatwość obsługi. Przyjazny interfejs sprawia że praca jest prosta i pozwala na osiągnięcie bardzo dobrych wyników.

\begin{figure}[H]
  \centering
    \includegraphics[width=100mm]{ge/gm_1.jpg}
  \caption{Google Maps.}
  \label{fig:googleMaps_1}
\end{figure}


\subsection{Windows Maps}
\label{subsec:Windows Maps}

Rysunek \ref{fig:bingMaps_1} przedstawia obraz otrzymany wykorzystując Bing Maps. W tym konkretnym przykładzie widzimy znaczną różnicę kolorów, obecność chmur pomniejsza wartość tych zdjęć. Należy wspomnieć o znacznie uboższym interfejsie dostarczanym użytkownikowi, interakcja jest w znacznym stopniu uboższa.

\begin{figure}[H]
  \centering
    \includegraphics[width=100mm]{ge/bing_1.jpg}
  \caption{Bing Maps.}
  \label{fig:bingMaps_1}
\end{figure}

\subsection{Yahoo Maps}
\label{subsec:Yahoo Maps}

Kolejnym dostarczycielem danych kartograficznym jest Yahoo, przykład znajduje się na rysunku \ref{fig:yahooMaps_1}. Interfejs jest zbliżony do Bing Maps, jednak obszar na którym możemy przedlądać zdjęcia jest mniejszy, obszary oddalone od większych miast nie są w pełni uwzglęnione.

tylko granice krakowa

\begin{figure}[H]
  \centering
    \includegraphics[width=100mm]{ge/yahoo_1.jpg}
  \caption{Yahoo Maps.}
  \label{fig:yahooMaps_1}
\end{figure}

\subsection{Apple Maps}
\label{subsec:Apple Maps}

Kolejną dużą marką która dostarcza informację jest Apple. Niestety nie ma wersji która pozwalałaby na dostęp do tej usługi z powszechnie używanych komputerów stacjonarnych.

\section{Google Earth}
\label{sec:Google Earth}

Ciekawe wykorzystanie obrazów satelitarnych i wskaźnika czasu zostało zaprezentowane w programie Google Earth. W aplikacji tej możemy zobaczyć nie tylko najbardziej aktualne zdjęcia, ale jesteśmy w stanie cofnąć się w czasie i zobaczyć jak wyglądał obszar na który patrzymy w przeszłości.

Przykład takiej sytuacji został przedstawiony na rysunku \ref{fig:lasVegas1}, obraz terenu na którym powstanie miasto Las Vegas w roku 1950. Jak teren ten wyglądał w roku 1977 widzimy na rysunku \ref{fig:lasVegas2}, pomimo widocznych zmian teren ten nadal w dużym stopniu jest pustynny,dopiero na rysunku \ref{fig:lasVegas3} widzimy aktualny stan miasta.

Dzięki funkcji zmiany punktu i kąta patrzenia, pokazywania ciekawych miejsc czy chociażby włączania trybu w którym budynki nabierają formy przestrzennej, 3D, możemy poprzez zabawę i wirtualne wycieczki poszeżać naszą wiedzę o otaczającym nas świecie.

\begin{figure}[H]
  \centering
    \includegraphics[width=100mm]{ge/01_1950.jpg}
  \caption{Las Vegas w 1950 roku.}
  \label{fig:lasVegas1}
\end{figure}

\begin{figure}[H]
  \centering
    \includegraphics[width=100mm]{ge/02_1977.jpg}
  \caption{Las Vegas w 1950 roku.}
  \label{fig:lasVegas2}
\end{figure}

\begin{figure}[H]
  \centering
    \includegraphics[width=100mm]{ge/03_2012.jpg}
  \caption{Las Vegas w 1950 roku.}
  \label{fig:lasVegas3}
\end{figure}

\section{Oś czasu}
\label{sec:osCzasu}

Jedną z większych uniedogonień podczas korzystania z map jest ich statyczność. Tradycyjne mapy pozwalają poznać jedynie jedną konkretną sytuację któa panowała na danym obszarze, nawet jeśli nałożymy kilka informacji z różnych okresów na jeden obraz, wynik może przestać być czytely. Korzystając w map dostępnych w internecie zazwyczaj istnieje opcja zmiany widoku. W jednej chwili możemy być na wysokości kilkuset kilometrów nad poziomem morza, aby w kolejnej znaleść się kilkaset metrów nad ziemią.

Ciekawą opcją jest dostarczana przez Google Earth możliwość zmiany czasu wykonania zdjęć. Niestety jest ona nieprzydatna jeśli chcemy przezentować własne dane.





\chapter{Opis rozwiazania}
\label{cha:problem}

Poniższy rozdział zawiera omówienie technologi które zostały wykorzystane w procesie tworzenia aplikacji, ze szczególnym uwzględnieniem funkcjonalności które okazały się najbardziej przydatne. Następnie zawarto szczegółowy opis sposóbów rozwiązania problemów które należało rozwiązać w trakcie tworzenia interaktywnej mapy.

\chapter{Opis rozwiązania}
\label{cha:Opis rozwiązania}

W poniższym rozdziale omówione zostaną kroki pracy.

\section{Transmisja danych}
\label{sec:transmisjaDanych}

Pierwszym aspektem który należy rozwiązać jest sposób przesyłania danych. Problem ten jest szczególnie istotny w omawianej pracy z uwagi na możliwość przesyłania dużej ilości informacji o granicach lub innych liniach przezentowanych na mapie. Do opisu kwadratowego obszaru wymagane jest przesłanie informacji o minimum 4 punktach. Jeżeli będziemy chcieli przekazać dokłądniejszy zarys obszaru, zaprezentować granicę państwa lub linię frotnu wojennego linia prosta w większości przypadków będzie zbyt ogólnym przybliżeniem, nie oddającym prawdziwej sytuacji.

Projekt zakłada korzystanie z lokalnej pamięci komputera użytkownika podczas pracy, jednak dostarczenie informacji, danych wejściowych nie powinno być ograniczone do wczytania ich z pliku tekstowego, należy pamiętać o możliwości przesłania ich z serwera aplikacji do użytkownika. W sytuacji takiej, gdzie w jednym momencie przesłane zostaja wszystkie informacje na któch będzie zachodzia praca, ilość informacji może być znaczna.

Z raportu Akamai wynika że śrenia przepływność łączy internetowych dla użytkowników korzystających z puli adresów IP przeznaczonych dla Polski w I kwartale 2012 r. wynosiła 5Mb/s  \underline{\texttt{http://www.rp.pl/artykul/924483.html}} (dostęp 13.04.2014). Jest to bardzo dobry wynik który plasuje Polskę w czołówce rankingu. Pomimo tego nie można pominąć faktu optymalizacji zapytać i danych przesyłanych, wymieniane dane pomiędzy użytkownikiem a serwerem powinny być jak najmniejsze. Duża popularność urządzeń mobilnych w których dostęp do internetu jest zapewniany często poprzez sieć bezprzewodową a dostęp do interentu nie jest jeszcze tak dogodny jak jest to w przypadku użytkowników stacjonarnych  wymusza optymalizację.

Kolejnym powodem dla którego odpowiedzi serwera powinny być jak najlżesze jest koszt pracy samego serwera. Jest to szczególnie widoczne w dużych aplikacjach mających wiele urzytkowników, czas jaki jest przeznaczany dla pojedyńczego użytkownika jest mnożony przez ich ilość. Z tego powodu zawsze podczas zwiększania ilości użtkowników korzystających z aplikacji następuje czas w którym należy zacząć korzystać z dodatkowego serwera. Celem programisty tworzącego kod który będzie wykorzystywał zasoby serwera(zarówno czas jak i pamięć) jest dbanie aby moment w którym niezbędne będzie korzystanie z większej ilośći maszym nastąpił przy jak największej ilości użytkowników.

xml - 729  557
json - 695  400


\section{Możliwości HTML5}
\label{sec:html5}
\nocite{xml50}
\nocite{proxml}
\nocite{pre1}
\nocite{pre2}
\nocite{googlemapsbegin}
\nocite{proHTML5}
HyperText Markup Language,  hipertekstowy język znaczników
Pozwala na opisanie struktury informacji zawartych na stronie internetowej, to dzięki niemu przeglądarka moze rozróżnić takie elementy jak hiperłącze, akapit czy chociażby nagłówek.

Podobnie jak w przypadku XML, tak i tutaj wymagane jest aby wykorzystywane znaczniki umieszczane były w nawiasach ostrokątnych, a każdy z nich miał swoje domknięcie.

Poprawnym zapisem jest <p>Wiadomość<\textbackslash p> który oznacza pojedyńczy akapit. Zapis <p>Wiadomość<p>, który różni się od poprzedniego brakiem znaku "\textbackslash" w drugim znaczniku, czyni to ten zapis niepoprawnym. Istnieje możliwość aby wykorzystać pojedyńczy znacznik, przykładem jest <br \textbackslash> określający wstawienie nowej lini w miejscu wystąpienia tagu.

Obecnie powszechnie używany standart HTML4 ma niestety wiele ograniczeń, z tego powodu pracowano nad jego następnikiem. 22 stycznia 2008 W3C opublikował HTML5, wtedy jeszcze jako jedynie szkic.

Bardzo ciekawym i wartym zainteresowania dodanym elementem w nowej wersji jest obecność znaczniku canvas. Pozwala on na dynamiczne , skryptowe renderowanie kształtów i obrazów. Dzięki temu obiektowi możliwe stało się tworzenie animacji czy nawet gier działających w przeglądarce bez konieczności używania dodatkowych wtyczek czy programów.

\underline{\texttt{http://techtrendy.pl/title,Pierwsza-gra-3D-napisana-w-HTML5,wid,14102779,wiadomosc.html?ticaid=6107dc}}

Przykładem wielkich możliwości jakie dostarcza udoskonalony język jest fakt iż już w roku 2011 powstała pierwsza trójwymiarowa gra stworzona w całości przy użyciu HTML5. Przykład grafiki widoczny jest na rysunku \ref{fig:html3d}.
Wcześniej takie efekty możliwe były jednynie przy wykorzystaniu technologi Flash, jej wadą była drudność edycji gotowego produktu, wymagało to specjalnego oprogramowania. Fakt tworzenia pliku wykonywalnego którego nie było możliwości edycji wymuszał dostęp do kodów źródłowych, czyli plików które tworzył autor programu.

\begin{figure}[H]
  \centering
    \includegraphics[width=100mm]{ge/html5_3d.jpg}
  \caption{Pierwsza gra 3D w html5.}
  \label{fig:html3d}
\end{figure}

Kolejną funkcjonalnością która wydaje się być przydatna w stosunku do omawianego projektu jest możliwość rysowania kształtów geomeycznych, również na innych obrazach. Przykładem wykorzystania tej technologi jest \ref{fig:canvas1} na którym przy pomocy okręgu zaznaczono rynek główny w Krakowie i jego okolice, możliwość zmiany przejżystości narysowanego kształtu pozwala aby obraz podnim był nadal widoczny.

  \begin{figure}[H]
  \centering
    \includegraphics[width=100mm]{ge/canvas1.jpg}
  \caption{P.}
  \label{fig:canvas1}
\end{figure}

\lstset{language=JavaScript}
\begin{lstlisting}[caption=json]

      var canvas = document.getElementById('myCanvas');
      var context = canvas.getContext('2d');
      var imageObj = new Image();
	
      imageObj.onload = function() {
      context.drawImage(imageObj, 69, 50);
	  context.beginPath();
      context.arc(canvas.width / 2, canvas.height / 2, 90, 0, 2 * Math.PI, false);
      context.fillStyle = "rgba(255, 0, 0, 0.5)";
      context.fill();
      context.stroke();
      };
      imageObj.src = './gm_1.jpg';

\end{lstlisting}

\subsection{Format zapisu}
\label{subsec:zapis}

Istnieje wiele formatów używanyc w zależności od konkretnych potrzeb. Można dokonać ogólnego podziału na dwa rodzaje. Pierwszym są pliki zapisane rastrowo, jest to zapis punktów o określonym kolorze i położeniu na płaszczyźnie. Efektem takiego podejścia jest pogorszenie obrazu w momencie dużego powiększenia. Przykładem są formaty takie jak ADRG,ASC czy RGB.
Porównanie litery a i jej 7-krotnego powiększenia w tym zapise zaprezentowano na rysunku \ref{fig:rast}
  \begin{figure}[H]
  \centering
    \includegraphics[width=50mm]{ge/a2.jpg}
  \caption{Zapis rastrowy}
  \label{fig:rast}
  \end{figure}

Innym podejściem jest zapis wektorowy. Jest to zapis oparty na formułach matematycznych m.in. proste i łuki. Dzięki takiemu podejściu obraz w dowolnej skali na identyczną jakość, widać to na rysunku \ref{fig:wekt}. Przykładowymi formatami są GML(Geography Markup Language) otwarty standart XML dla wymiany danych GIS,Shapefile czy tradycyjny kartezjański układ współrzędnych.

  \begin{figure}[H]
  \centering
    \includegraphics[width=50mm]{ge/a1.jpg}
  \caption{Zapis wektorowy}
  \label{fig:wekt}
  \end{figure}

Do celów projektu zdecydowano się skorzystać z zapisu wektorowego. Oprócz bez problemowej skalowalności, pozwala on również na szeroką i łatwą edycję. W prosty sposób można dodać informacje nie tylko o położeniu, ale inne takie jak słowny opis miejsca lub wydarzenia które miało miejsce w pobliskim rejonie.

Format KML jest używany do reprezentacji danych graficznych w aplikacjach firmy Google, takich jak Google Maps czy Google Earth. Oparty jest na standardzie XML, wrażliwy na wielkość liter, wymusza ścisłą poprawność danych z formatem. Oznacza to że każdy tag, element ma ścisle określone miejsce i nie może pojawić się nigdzie indziej.

Aby tworzona aplikacja miała możliwość współpracy ze stworzonymi wczesniej zbiorami danych, mapami mimo faktu iż informacje te nie będą w pełni wykorzystywały możliwości frameworku, stworzony został parser plików kml. Dzięki temu możliwy jest import danych, uzupełnienie ich o dodatkowe informacje i zapis do własnego formatu.

\section{Parser plików}
\label{sec:parser}

Założenia projektu zapewniają dwa źródła dostarczania informacji niezbędnych do poprawnej pracy frameworka.
Pierwszym jest przesyłanie infromacji z serwera do urządzenia klienta przy wykorzystaniu formatu JSON, wiadomość taka jest zbudowana w odpowiednią strukturę aby bezpośrednio dokonać konwersji do jegnego ciągu znakowego. W ten sposób możliwe będzie zapewnienie w bardzo szybkim czasie danych pozwalających na pełne wykorzystanie wszystkich możliwości aplikacji.

Dodatkową opcją dostarczania informacji będzie wczytywanie pliku z informacjami opisującymi mapę. Do tego celu stworzono parser plików który pozwala na dokonanie analizy wczytanego pliku i konwersję danych do formatu używanego do przechowywania danych po stronie klienta.

\section{Oś czasu}
\label{sec:timeLine}

Tematem pracy oprócz wykorzystania i prezentacji danych na bazie informacji geograficznych jest ich połączenie z osią czasu, która pozwoliłaby na zmianę prezentowanych danych ze względu na interesujący nas okres czasu.

Ponieważ własna implementacja osi czasu która nieograniczałaby użytkownika, pozwala mu w pełni korzystać z możliwości pełnej aplikacji, która tylko w części składa się z możliwości manipulacji czasem postanowiono wykorzystać istniejące rozwiązanie.

Przeprowadzając badania rozwiązań głównymi aspektami na które zwrócono uwagę było:

\begin{itemize}

\item

Typ licencji
Niezwykle ważne jest aby końcowy projekt był całkowicie otwarty dla dalszego rozwoju, pozwalał użytkownikom na adaptację rozwiązań dla swoich potrzeb. Z uwagi na to licencja musi pozwalać na nieograniczoną modyfikację kodu źródłowego.

\item

Wykorzystywane technologie
Pamiętając że efektem końcowym powinien być framework odpowiadający szerokiemu gronowi odbiorców, będący jednocześnie darmowym ważne aby języki programowania wykorzystane przy jego tworzeniu też taki były. Oznacza to nie korzystanie z płatnych, wymagających licencji oprogramowań. Używane standarty powinny być rozpropagowane i używane przez szerokie grono obiorców.

\end{itemize}

Po długich i wnikliwych poszukiwaniach zdecydowano skorzystać z widget-u o nazwie Timeline który został stworzony przej projekt SIMILE działający na uczelni MIT. Pozwala on na bardzo dużą konfigurowalność dzięki czemu jego modyfikacja, nawet bez ingerencji w kod źródłowy jest bardzo prosta.
Projekt jest w stanie stworzyć kilka pasm któe będą określały interwały czasu. Pozwala to aby wynik końcowy był przejżysty bez względu czy prezentujemy wydarzenia które miały miejsce w okresie kilku minut czy setek lat.

Problemem który pojawił się było połączenie wybranego sposobu prezentacji osi czasu z mapą i elementami które powinny zmieniać się wraz z upływem czasu.

Rysunek \ref{fig:tm1} prezentuje możliwości tego projektu. Pasmo odpowiadają zakresom czasu, nie muszą on jednak być jednakowe w każdym miejscu. W zaprezentowanym przykładie dzień 23 listpoada uznany został za warty dokładniejszemu przyjżeniu,łatwo możemy z dokładnością co do godziny zmieniać zakres czasu.Natomiast kolejny miesiąc może być o wiele mniej ciekawy, a obszar przeznaczony dla niego być mniejszy niż ten posiadany przez wymieniony powyżej dzień.

Dla porównania \ref{fig:tm2} prezentuje o wiele prostszą konfigurację w której dolne pasmo podzielone jest przez miesiące, natomiast góre przez tygodnie.

  \begin{figure}[H]
  \centering
    \includegraphics[width=150mm]{ge/tm1.jpg}
  \caption{Timeline}
  \label{fig:tm1}
\end{figure}

  \begin{figure}[H]
  \centering
    \includegraphics[width=150mm]{ge/tm2.jpg}
  \caption{Timeline 2}
  \label{fig:tm2}
\end{figure}

\section{Interferjs}
\label{sec:Interferjs}

Aby tworzone rozwiązanie mogło być używane przez szerokie grono odbiorców niezbędnym elementem jest stworznie interfejsu który pozwlałby na edycję danych wczytanych z pliku, lecz także pozwalał na stworzenie nowej mapy. W tym celu stworzono interfejs graficzny do edycji danych.

\chapter{Przykład użycia}
\label{sec:useexample}
10 %
Poniższy rozdział zawiera przykład użycia stworzonej aplikacji. Prezentuje jedynie niewielką część możliwości.


\section{Historia USA}
\label{sec:usahistory}

W celu zobrazowania powstawania USA poniżej zaprezentowano 3 stany jego rozwoju. Na rysunku \ref{fig:st1} znajduje się obszar państwa w momencie ratyfikacji konstytucji.Kolorem czarnym zaznaczone są 13 pierwszych stanów które uznawane są za założycielskie. Pokazano również możliwość dodawania opisu do wydażeń dodanych do sceny, umożliwia to dodanie dodatkowych informacji które mogą wspomóc proces dydatktyczny.

\begin{figure}[H]
  \centering
    \includegraphics[width=100mm]{ge/st1.jpg}
  \caption{rok.}
  \label{fig:st1}
\end{figure}

Kolejny rysunek \ref{fig:st2} opisuje stan z 1912 roku. Dzięki możliwości stworzenia dwuch lub pasm które będą odpowiadały za zmianę daty, przesunięcie się o ponad 100 lat nie sprawia większego trudu i możliwe w krótkim czasie.

\begin{figure}[H]
  \centering
    \includegraphics[width=100mm]{ge/st2.jpg}
  \caption{rok.}
  \label{fig:st2}
\end{figure}

Ostani element tej serii ukazuje aktualny stan granic. Widać że utrzymuje się on od 1959 roku.

\begin{figure}[H]
  \centering
    \includegraphics[width=100mm]{ge/st3.jpg}
  \caption{rok.}
  \label{fig:st3}
\end{figure}

\chapter{Podsumowanie}

2-3 strony





% itd.
% \appendix
% \chapter{Dodatek A}
\label{cha:appa}

\section{Oryginalny plik kml}
\label{sec:akml}

\lstset{language=XML}
\begin{lstlisting}[caption=Plik KML]
<?xml version="1.0" encoding="UTF-8"?>
<kml>
  <Document>
    <name><![CDATA[US States]]></name>
    <open>1</open>
    <Style id="Style_5">
      <LabelStyle>
        <color>9900ffff</color>
        <scale>1</scale>
      </LabelStyle>
      <LineStyle>
        <color>990000ff</color>
        <width>2</width>
      </LineStyle>
      <PolyStyle>
        <color>997f7fff</color>
        <fill>1</fill>
        <outline>1</outline>
      </PolyStyle>
    </Style>
    <Placemark id="pm251">
      <name><![CDATA[Hawaii (1959)]]></name>
      <Snippet maxLines="0">empty</Snippet>
      <description></description>
      <TimeSpan>
        <begin>1989</begin>
      </TimeSpan>
      <styleUrl>#Style_5</styleUrl>
      <MultiGeometry>
        <MultiGeometry>
          <Polygon id="g867">
            <altitudeMode>clampToGround</altitudeMode>
            <outerBoundaryIs>
              <LinearRing>
                <coordinates>
-159.335174733889,21.9483433404175
-159.327130348878,22.0446395507162
-159.295025589769,22.1248124949548
-159.343195828355,22.1970166285359
-159.391366885913,22.2291198667724
-159.576012589057,22.2131796383001
-159.712505933171,22.1490592515515
-159.800814224332,22.0366665967853
-159.736592652746,21.9644203111023
-159.640246973766,21.9483657695954
-159.576021285803,21.8841361312636
-159.439545188912,21.8680716835921
-159.335174733889,21.9483433404175
                </coordinates>
              </LinearRing>
            </outerBoundaryIs>
          </Polygon>
        </MultiGeometry>
      </MultiGeometry>
    </Placemark>
  </Document>
</kml>

\end{lstlisting}

\section{Zapis w pamięci sesyjnej}
\label{sec:ass}

\lstset{language=XML}
\begin{lstlisting}[caption=Zapis w Storage]
[{"start":"1989","end":"2010","polygon":[{"lat":21.9483433404175,"lon":-159.335174733889},{"lat":22.0446395507162,"lon":-159.32713034887797},{"lat":22.1248124949548,"lon":-159.295025589769},{"lat":22.1970166285359,"lon":-159.34319582835496},{"lat":22.2291198667724,"lon":-159.391366885913},{"lat":22.2131796383001,"lon":-159.57601258905697},{"lat":22.1490592515515,"lon":-159.71250593317097},{"lat":22.0366665967853,"lon":-159.800814224332},{"lat":21.9644203111023,"lon":-159.73659265274603},{"lat":21.9483657695954,"lon":-159.640246973766},{"lat":21.8841361312636,"lon":-159.576021285803},{"lat":21.8680716835921,"lon":-159.439545188912},{"lat":21.9483433404175,"lon":-159.335174733889}],"title":"Hawaii (1959)","mId":"pm251","pId":"g867","color":"#ff0000","width":2,"opacity":0.59765625,"fillcolor":"#ff7f7f","name":"Hawaii (1959)","desc":""}]
\end{lstlisting}

\section{Wynik końcowy}
\label{sec:aresult}

\begin{figure}[H]
  \centering
    \includegraphics[width=100mm]{ge/hawaii.jpg}
  \caption{Hawaje}
  \label{fig:hawaii}
\end{figure}

% \include{dodatekB}
% itd.

\bibliographystyle{alpha}
\bibliography{bibliografia}
%\begin{thebibliography}{1}
%
%\bibitem{Dil00}
%A.~Diller.
%\newblock {\em LaTeX wiersz po wierszu}.
%\newblock Wydawnictwo Helion, Gliwice, 2000.
%
%\bibitem{Lam92}
%L.~Lamport.
%\newblock {\em LaTeX system przygotowywania dokumentów}.
%\newblock Wydawnictwo Ariel, Krakow, 1992.
%
%\bibitem{Alvis2011}
%M.~Szpyrka.
%\newblock {\em {On Line Alvis Manual}}.
%\newblock AGH University of Science and Technology, 2011.cccccc
%\newblock \\\texttt{http://fm.ia.agh.edu.pl/alvis:manual}.
%
%\end{thebibliography}

\end{document}
