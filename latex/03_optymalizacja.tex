\section{Optymalizacja rozwiązania}
\label{sec:optymalizacja}

\subsection{Funkcje haszujące}
\label{sec:hashfunction}

\subsection{Wydajność parsera}
\label{subsec:wydajnosc}

W celach sprawdzenia wydajności zaimplementowanego parsera przeprowadzono testy porównawcze. W pierwszej próbie wykorzystano dwa pliki zawierające dane w formacie który pozwolił na jego analizę, pierwszy "plik1.kml" składał się z jednego obszaru i jednego stylu określającego preferencje graficzne, jego dokładana zawartość została zawarta w załączniku \ref{sec:akml}. Drugi plik "plik2.kml" zawierał informacje o granicy wszystkich stanów USA, na potrzeby badań on również zawierał informacje o jednym stylu.
Wykonano proces wczytania ich zawartości, mierząc za każdym razem czas potrzebny na zakończenie procesu. Wyniki zamieszczono w zbiorczej tabeli \ref{tab:speedTest}.

Dokładne informacje dotyczące zawartości plików zamieszczono w tabeli \ref{tab:testFile}

\begin{table}[H]
    \centering
    \begin{tabular}{|l|l|l|}
    \hline
    Nazwa pliku & Ilość punktów & Ilość poligonów \\ \hline
    plik1.kml & 13 & 1 \\ \hline
    plik2.kml & 13697 & 133 \\ \hline

    \end{tabular}
    \caption{Pliki testowe}
    \label{tab:testFile}
\end{table}


\begin{table} [H]
    \centering
    \begin{tabular}{|l|l|l|}
    \hline
    Test & 13 punktów & 12697 punktów \\\hline
    I & 3 & 343 \\\hline
    II & 13 & 449 \\\hline

    \end{tabular}
    \caption{Czas dostępu [ms]}
    \label{tab:speedTest}
\end{table}
