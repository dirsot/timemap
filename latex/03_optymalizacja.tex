\clearpage
\newpage
\section{Optymalizacja rozwiązania}
\label{sec:optymalizacja}
 

\subsection{Wydajność parsera}
\label{subsec:wydajnosc}

W celach sprawdzenia wydajności zaimplementowanego parsera przeprowadzono testy porównawcze. W pierwszej próbie wykorzystano dwa pliki zawierające dane w formacie który pozwolił na jego analizę, pierwszy ``plik1.kml''  składał się z jednego obszaru i jednego stylu określającego preferencje graficzne, jego dokładana zawartość została zawarta w załączniku \ref{sec:akml}. Drugi plik ``plik2.kml'' zawierał informacje o granicy wszystkich stanów USA, na potrzeby badań on również zawierał informacje o jednym stylu.
Wykonano proces wczytania ich zawartości, mierząc za każdym razem czas potrzebny na zakończenie procesu. Wyniki zamieszczono w zbiorczej tabeli \ref{tab:speedTest}.

Dokładne informacje dotyczące zawartości plików zamieszczono w tabeli \ref{tab:testFile}

\begin{table}[H]
    \centering
    \begin{tabular}{|l|l|l|}
    \hline
    Nazwa pliku & Ilość punktów & Ilość poligonów \\ \hline
    plik1.kml & 13 & 1 \\ \hline
    plik2.kml & 13697 & 133 \\ \hline

    \end{tabular}
    \caption{Pliki testowe}
    \label{tab:testFile}
\end{table}


\begin{table} [H]
    \centering
    \begin{tabular}{|l|l|l|}
    \hline
    Test & 13 punktów & 12697 punktów \\\hline
    I & 3 & 343 \\\hline
    II & 13 & 449 \\\hline

    \end{tabular}
    \caption{Czas dostępu [ms]}
    \label{tab:speedTest}
\end{table}

\subsection{Reprezentacja graficzna}
\label{subsec:showing}

W celu optymalizacji działania aplikacji zastosowano kilka rozwiązań mających na celu przyścieszenie działania. Dotyczą one sposobu w jaki aplikacja renderuje graficzna częśc projektu.


\begin{description}
\item[Stopniowa generacja obiektów]\hfill \\
Nie wszystkie obiekty dotyczące mapy są renderowane podczas jej inicjalizacji. Wstępna konfiguracja pozwala na stworzenie mapy z osią czasu nie pokazującą całego przedziału czasowego do którego odwołują się informacje wejściowe, możliwe jest ustawienie aby na ekranie pojawiły się informacje dotyczące wydarzeń z jednoego dnia podczas gdy dane mogą zawierać informacje z okresu kilku lat. Stworzony algorytm można przedstwić w najstępujących krokach.

\begin{enumerate}
\item
Renderowanie jedynie elementów widocznych na ekranie (mających miejsce w widocznym okresie czasu).

\item
Renderowanie dodatkowych elementów zawierających się w otoczeniu wynoszącym 10\% aktualnego okresu a następnie ukrycie ich. Takie rozwiązanie pozwala, w większości przypadków, na rederowanie nawet dużych i skomplikowanych obiektów bez wpływu na działanie użytkownika(posiada on dane wygenerowane w pkt. 1 i nie oczekuje na wygenerowanie kolejnych).

\item
Podczas przesunięcia lini czasu w pierwszym momencie elementom w najbliższym sąsiedztwie ustawiana jest flaga widoczności, jeśli zmiana okresu była niewielka akcja ta jest wystarczająca aby pokazać wszystkie wyamgane elementy, w przeciwnym przypadku nastepuje generacja brakujacych. W obu przypadkach po zakończniu powtarzay jest pkt 2. Obiekty które nie powinny być widoczne nie są niszczone jedynie chowane.

\end{enumerate}

\item[różne poziomy test]\hfill \\

\end{description}
